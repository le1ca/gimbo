\documentclass[11pt]{article}

\usepackage{microtype}

\title{\textbf{Markov chain text generation with inferred parts of speech}}
\author{Travis Mick}

\begin{document}

\maketitle

\begin{abstract}
%%
Markov chains have commonly been used in text generators for applications
such as chat bots, typically implemented by utilizing the likelihood of $k$-word
sequences to inform the iterative production of words.
%
In this document, we outline a technique to improve the quality of such systems'
outputs by leveraging inferred knowledge about the parts of speech for each
word in a sequence.
%
\end{abstract}

\section{Introduction}

%%
Traditional Markov chaining text generation leverages a database indexing the
frequency $F_W(w)$ of each observed $k$-word sequence $w = (w_0, w_1, \dots, w_{k-1})$.
%
The database can be queried for a likely sequence given some prefix.
%
Given a prefix $w' = (w_0, w_1, \dots, w_{j-1})$ of length $j < k$, a candidate
sequence $c$ with prefix $w'$ can be assigned a score as
%
\begin{equation}
\label{eqn:wordscore}
S_W(c) = \frac{F_W(c)}{\sum_{c' \in C_W(w')} F_W(c')}
\end{equation}
where $C_W(w')$ is the set of all possible sequences with the prefix $w'$.
%
Ostensibly, the highest scored sequence is returned by the text generation algorithm,
and an output string can be obtained by iterating this process while maintaining a
rolling window of $k$ words.
%

%%
Unfortunately, the choice of $k$ introduces a trade-off between overfitting and underfitting.
%
Too low of $k$ would result in unnatural sequences of words being produced, and a $k$ too large
would yield few novel results.
%
Therefore, we suggest augmenting the $k$-word Markov chain with additional information about
the parts of speech of the constituent words to increase output quality given small $k$.
%

\section{Generation model}

%%
In addition to a word-sequence frequency database, we will utilize a database of frequencies for
part of speech sequences, as well as a database of frequencies of observations of a word being
used as a particular part of speech.
%

%%
First, let $F_P(p)$ give the frequency of observations of $k$-length part of speech sequences
$p = (p_0, p_1, \dots, {p_{k-1}})$.
%
Then, let $P(w_i, p_i)$ give the frequency of observations of word $w_i$ used as
part of speech $p_i$.
%
We can now define the likelihood of a word sequence $w$ corresponding to a part of speech
sequence $p$ as
%
\begin{equation}
\label{eqn:probcorr}
P_A(w, p) =  \frac{F_P(p)}{\sum_{p' \in X^k} F_P(p')} \cdot \prod_{0 \leq i < k} \frac{P(w_i, p_i)}{\sum_{p'_i \in X} P(w_i, p'_i) }
\end{equation}
%
Where $X$ is the set of all parts of speech and $X^k$ is the set of all possible part of speech
sequences of length $k$.
%
Note that we have utilized information about the correspondence between parts of speech
and words, as well as the likelihood of the part of speech sequence itself.
%
We can leverage this to produce an augmented score for a candidate word sequence $c$
with prefix $w'$.
%
\begin{equation}
\label{eqn:augscore}
S_A(c) = S_W(c) \cdot \max_{p \in X^k} \left( P_A(c, p) \right)
\end{equation}
%
Note that we score the word sequence based on the best possible part of speech assignment
as informed by Eqn.~\ref{eqn:probcorr}.
%

\section{Learning model}

%%
In many applications, Markov chain text generators attempt to learn from observations in
real time.
%
While the learning process to inform $F_W$ is straightforward, we must define a mechanism
to label observed word sequences with parts of speech in order to update $F_P$.

%%
When attempting to assign a part of speech sequence to a word sequence, we must come up with
a scoring mechanism to implement inference, as was done with word sequences in
Eqn.~\ref{eqn:wordscore}.
%
Given a part of speech sequence prefix $p' = (p_0, p_1, \dots, p_{j-1})$ of length $j < k$
and a corresponding word sequence $w$, a candidate part of speech sequence $c$ with prefix $p'$
can be assigned a score as 
%
\begin{equation}
\label{eqn:posscore}
S_P(c) = P_A(w, c) \cdot \frac{F_P(c)}{\sum_{c' \in C_P(p')} F_P(c')}
\end{equation}
%
where $C_P(p')$ is the set of all possible sequences with the prefix $p'$.
%
We have again leveraged Eqn.~\ref{eqn:probcorr} to provide weight to both observed
part of speech sequences and the correspondence between parts of speech and words.
%
Note that because our learning model requires prior observations in order to infer
parts of speech for new observations, the system must be trained prior to being
exposed to input.
%

\end{document}
